% !TEX root = FacMik.tex

\section{A characterization of regular languages of Wadge degree bigger or equal $\omega_1$}
\subsubsection*{Games on types and strategy trees.}
%We recall that, given a language $L$, the set $H_L$ is the set of tree types, and the set $V_L$ of context types of $L$. Both are finite when $L$ is regular. 
Following \cite{bp}, for a given regular language of trees $L$, a prefix $p$ and types $h_i\in H_L$ ($i=1,2,\dots$) we define games on types $\mathcal{H}^p_k(h_1,\dots,h_k)$ and $\mathcal{H}^p_\infty(h_1,h_2,\dots)$. The Constrainer plays as before and the task of the Alternator is to play in the $i$--th round a tree of type $h_i$, that is an element of $\alpha_L^{-1}(h_i)$. If the prefix $p$ is empty we just write $\mathcal{H}_\infty(h_1,h_2,\dots)$ and $\mathcal{H}(h_1, \dots, h_k)$ respectively.


A type tree for $L$ is a tree over the finite alphabet $H_L$. For a given tree $t$, there is a type tree $\sigma_t$ induced by $t$ such that for every node $w \in \dom(\sigma_t)$, $\sigma_t(w)$ is the type of the tree $t.w$.
Let $\sigma$ be a type tree, and $t$ a tree. We say that a type tree $\sigma$ is \emph{locally consistent} with a tree $t$ if $\dom(\sigma)=\dom(t)$ and for every node $w \in \dom(t)$ such that $t(w)=a$, 
\begin{itemize}
\item if $w$ is a leaf, then $\sigma(w)$ is the type of $a$,
\item if $w$ has two children $m_\ell$ and $m_r$, then $\sigma(w)$ is the type obtained by applying $a$ to the pair $(\sigma(m_\ell), \sigma(m_r))$.
\end{itemize}

\begin{definition} A finite strategy tree is a tuple
$\mathfrak{s}=(t, \sigma_1, \dots, \sigma_k)$ where
\begin{itemize}
\item $t$ is a tree, the support of the strategy
\item $\sigma_t=\sigma_1$, and
\item $\sigma_\ell$ is locally consistent with $t$, for each $\ell \leq k$,
\item for each $w \in \dom(t)$, Alternator has a winning strategy in $\mathcal{H}(\sigma_1(w), \dots, \sigma_k(w))$.
\end{itemize}
An  infinite strategy tree is defined analogously.
\end{definition}
The root sequence of a strategy tree $\mathfrak{s}=(t, (\sigma_i : i < \alpha))$ is the sequence of types $(\sigma_i(\varepsilon): i < \alpha)$.

Let $\mathfrak{s}$ be a finite strategy tree. The \emph{root alternation} of $\mathfrak{s}$ is the alternation of the root sequence, while the limit alternation of $\mathfrak{s}$ is the maximal number $k$ such that infinitely many subtrees of $\mathfrak{s}$ have root alternation $k$. We say that a set  $\mathfrak{S}$ of finite strategy trees has \emph{bounded root alternation} if there is a  $k$ such that the root alternation of each $\mathfrak{s} \in \mathfrak{S}$ is at most $k$, unbounded otherwise. Analogously for limit alternation.

A finite or infinite strategy tree $\mathfrak{s}=(t, \sigma_1, \dots)$ is \emph{locally optimal} if for every strategy tree $\mathfrak{s}'=(t, \sigma'_1, \dots)$ with same root sequence, and every $i>1$, the depth at which $\sigma_i$ and $\sigma_{i+1}$ first differ is greater or equal than the depth at which $\sigma_i$ and $\sigma'_{i+1}$ first differs.
The next Proposition is the most important technical point of \cite{bp}. 



\begin{proposition}[\cite{bp}]\label{prop:locality}
 For a regular tree language $L$, if $\mathfrak{S}$ is a set of locally optimal finite strategy trees with both root and limit unbounded alternation, then the strategy graph $G_L$ is recursive.
 \end{proposition}
 
The next proposition will be important in the rest of the paper. It establishes a link between infinite cutting games alternating between a language and its complement and infinite cutting games with infinite root alternation.

\begin{proposition}\label{prop:types}
 Assume Alternator has a winning strategy in  $\mathcal{H}^\varepsilon_\infty(L, L^\complement)$. Then there is an infinite strategy tree $\mathfrak{s}^\infty$ with infinite root alternation. 
\end{proposition} 
 \begin{proof}
 Assume Alternator has a winning strategy $f$ in  $\mathcal{H}^\varepsilon_\infty(L, L^\complement)$. The infinite strategy tree $\mathfrak{s}^\infty$ is constructed as follows. First of all, we can represent $f$ as a tree satisfying the following properties:
\begin{itemize}
\item the root is labelled by $\varepsilon$, and its unique child is labelled by Alternator's move obtained by applying the winning strategy $f$ at the first round of the game,
\item if a node $n$ is labelled with a tree $t$, then for every prefix $p$ of $t$ there is a unique child of $n$ labelled it by $p$,
\item if a node $n$ is labelled with a prefix $p$, then $n$ has a unique child, and such a child is labelled by the answer obtained by applying the winning strategy $f$ to the position in the cutting game given by the  labels of the path from the root to $n$.
\end{itemize}
From the previous construction, we have that nodes at odd depth represents Alternator's move and are therefore labelled by trees, while nodes at even depth represents Constrainer's move and are thus labelled by prefixes. From now on, we always identify $f$ and the aforementioned  tree.


Given a node $n$ of $f$ labelled by a prefix $p$, and $\alpha < \omega$,  by $(f.n)|_k$, we denote the induced  winning strategy  for Alternator in  $\mathcal{H}^p_\alpha(L, L^\complement)$, if $n$ is at depth $2i$ with $i$ even, and in  $\mathcal{H}^p_\alpha(L^\complement, L)$, if $n$ is at depth $2i$ with $i$ odd.
Now, if we prove the following claim we are done:
\begin{claim}\label{claim:strategy}
For every node $n$ of $f$ labelled by a prefix $p$, 
there is a sequence of strategy trees $\mathfrak{S}_n=(\mathfrak{s}^n_i: i < \omega)$ such that for each $n$
\begin{enumerate}
\item $\mathfrak{s}^n_\ell=(t, \sigma_1, \dots, \sigma_\ell)$, with the type $\sigma_{2k+1}(\varepsilon)$ included in $L$ and the type  $\sigma_{2k}(\varepsilon)$ included  $L^\complement$ if $n$ is at depth $2i$ with $i$ even, else dually. In particular this means that  $\sigma_{2k+1}(\varepsilon) \neq \sigma_{2k}(\varepsilon)$;
\item $\mathfrak{s}^n_{\ell+1}$ extends $\mathfrak{s}^n_\ell$, that is  $\mathfrak{s}^n_{\ell+1}=(t, \sigma_1, \dots, \sigma_\ell, \sigma_{\ell+1})$ and $\mathfrak{s}^n_\ell=(t, \sigma_1, \dots, \sigma_\ell)$.
\end{enumerate}
\end{claim}
Indeed, given the claim above, from point 1 we have that
for each node $n$ labelled by a prefix $p$, and each $k < \omega$, $\mathfrak{s}^n_k$ has root alternation $k$ and defines a winning strategy for Alternator in $\mathcal{H}^p_k(L, L^\complement)$ if $n$ is at depth $2i$ with $i$ even, in $\mathcal{H}^p_k(L^\complement, L)$ else.
From point 2, let $\mathfrak{s}^\varepsilon_k = (t, \sigma_1, \dots, \sigma_k)$ for $k=1,2,\dots$. We define  
the required $\mathfrak{s}^\infty = (t, \sigma_1, \dots )$.

The Claim  is proved as follows. 
Firstly, we describe an inductive procedure that for each $\ell$ assign to each node $n$ of $f$ labelled by a prefix a strategy tree $\mathfrak{s}^n_\ell$.
We then verify that for each node $n$ of $f$, the obtained sequence $\mathfrak{S}_n=(\mathfrak{s}^n_i: i < \omega)$ satisfy
the properties of the Claim. 

We start by describing the procedure, and already verify that it satisfies property 1 of the claim. First notice that given an infinite sequence of type trees $(\sigma_1, \dots)$, by compactness there is a converging subsequence $(\sigma'_1, \dots)$. We assume that every time we have to choose  
a converging subsequence $(\sigma'_1, \dots)$ of a given sequence $(\sigma_1, \dots)$, we always choose the same sequence. We also assume that given a tree $t$, we have fixed an enumeration $(p_1, \dots)$ of all its prefixes.
For $k=1$, 
it is enough to take for each node $n$ and prefix label $p$, $\mathfrak{s}^n_1=(t, \sigma_t)$, where $t$ is given by applying $f$ to the considered position. By choice of $\sigma_t$, Property 1 is satisfied.
For $k>1$ we proceed as follows. We assume the construction performed for $k-1$. Fix any node $n$ labelled by a prefix $p$. Assume the answer given by $f$  at the position in the game given by the path to the root to node $n$ is $t$. To every prefix $p$ of $t$ corresponds a child $m$ of $n$ to which we already associated a strategy tree $\mathfrak{s}^m_{k-1}=(t^p, \sigma^p_{2}, \dots, \sigma^p_{k})$. 
Let us thence consider the sequence $(p_1, \dots)$, with limit $t$ and the sequences
$(t^{p_1}, \dots)$, $(\sigma^{p_1}_2\dots)$, $\dots,$  $(\sigma^{p_1}_{k}\dots)$.  The limits of these sequences were chosen in advance and are $t^*, \sigma^*_2, \dots, \sigma^*_k$. Since each $t^{p_i}$ extends $p_i$, the limit $t^*$ of  $(t^{p_1}\dots)$ is $t$.
Now, for each $p$, the type trees $(\sigma^p_{2}, \dots, \sigma^p_{k})$
are locally consistent with $t^p$. Furthermore, given a sequence of trees $(t_1, \dots)$  that converges to $t^*$ and a sequence of type trees $(\sigma_1, \dots)$  that converges to $\sigma^*$, if $\sigma_n$ is locally consistent with $t_n$ for every n, then $\sigma^*$ is locally consistent with $t^*$.
%Therefore, by Lemma \ref{lemma:limit} 
From this fact follows that the limits $\sigma^*_2, \dots, \sigma^*_k$ are locally consistent with $t$. Finally, define $\sigma^*_1$ to simply be $\sigma_t$. We have just proved that
$\mathfrak{s}^n_k = (t, \sigma^*_1, \dots, \sigma^*_k)$
is a strategy tree. From induction hypothesis together with definition of $\sigma_t$ and preservation of Property 1 under limits follows that $\mathfrak{s}^n_k$ also 
satisfies Property 1. 

We now verify that the described procedure preserves Property 2. For $k=1$ there is nothing to check. For the induction step, we reason as follows. Assume the Property holds for each node and for each $i<k$. Now, let us consider an arbitrary node $n$. We have to prove that $\mathfrak{s}^n_{k+1}$ extends $\mathfrak{s}^n_{k}$. 
By induction hypothesis,  $\mathfrak{s}^m_{k-1}=(t^p, \sigma^p_{2}, \dots, \sigma^p_{k})$ and
$\mathfrak{s}^m_{k}=(t^p, \sigma^p_{2}, \dots, \sigma^p_{k}, \sigma^p_{k+1})$, for every node $m$  in the described procedure. Since the limits have been fixed in advanced, we have that $\mathfrak{s}^n_{k}=(t, \sigma_t, \sigma^*_2, \dots, \sigma^*_k)$ and $\mathfrak{s}^n_{k+1}=(t, \sigma_t, \sigma^*_2, \dots, \sigma^*_k, \sigma^*_{k+1})$, meaning that the latter extends the former. This concludes the proof of the claim.
 \end{proof}
 
Using the above Proposition, we can generalize to infinite games Proposition 5.2 from \cite{bp}:
\begin{proposition}\label{prop:tree_to_types}
 For a regular language $L$ of infinite trees and any prefix $p$, the following conditions are equivalent.
 \begin{enumerate}
\item Alternator wins the game $\mathcal{H}^p_\infty(L, L^\complement)$, 
 \item  There are tree types $h, g\in H_L$, such that $h\neq g$ and Alternator wins $\mathcal{H}^p_\infty(h, g)$.
 \end{enumerate}
\end{proposition}
The proof of Proposition \ref{prop:tree_to_types} can be found in the Appendix.
%\begin{proof}
%The direction from (1) to (2) is an immediate corollary of Proposition \ref{prop:types}.
%For the direction from (2) to (1) we reason as follows. Since $g$ and $h$ are different elements of the syntactic algebra, it follows that there must be some multi-context $c$ such that the tree type $c[g]$ is contained in $L$, while the tree type $c[h]$ is disjoint with $L$. 
%This means that Alternator has a winning strategy in the game $\mathcal{H}^p_\infty(L, L^\complement)$.
%\end{proof}

The following Lemma, proved in \cite{bp} for finite strategy trees but whose proof extends straightforwardly to infinite strategy trees too, will turn out to be useful.%\footnote{This Lemma is . CHECK}

\begin{lemma}\label{lemma:locallyoptimal}
For every finite or infinite strategy tree, there is a locally optimal strategy tree with same root sequence.
\end{lemma}
%\begin{proof} Let $\mathfrak{s}=(t, (\sigma_i : i < \alpha))$ be a strategy tree. We construct the locally optimal strategy tree $\mathfrak{s}=(t, (\sigma'_i : i < \alpha))$ by induction as follows. We put first $\sigma'_1=\sigma_1$. Then, consider the set $\mathfrak{S}_2$ of all strategy trees (finite or infinite) that are locally consistent with $t$ and which have the same root value as $\sigma_2$. This set is a closed set, and therefore is compact. It follows that some element of this set minimizes the distance with respect to $\sigma_2$. Such element will be the new $\sigma'_2$. We proceed likewise for the remaining coordinates $j$, with $2 < j < \alpha$.
%\end{proof}
 The next Lemma  follows immediately from the definition of a strategy tree.
\begin{lemma}\label{lemma:short_strategy}
Let $\mathfrak{s}=(t, \sigma_1, \dots, \sigma_k)$ be a strategy tree. For the game $\mathcal{H}(\sigma_1(\varepsilon), \dots, \sigma_k(\varepsilon))$ and a strategy of Constrainer given by always cutting at level $i$, Alternator wins  by playing as follows:
\begin{itemize}
\item at first, Alternator plays $t$, then
\item for each port $w$ at level $i$ of the multi context given by Constrainer's move, Alternator plugs in the tree given by her winning strategy $\mathcal{H}(\sigma_1(w), \dots, \sigma_k(w))$.
\end{itemize}
In particular, if from a certain $j<k$ on $\sigma_\ell(w)=\sigma_{\ell+1}(w)$, $j\leq \ell < k$, then for each round $j< \ell < k$ she always plugs in the same tree of type $\sigma_j(w)$ chosen at round $j$.
\end{lemma}





%The following Lemma will turn out to be useful:
%\begin{lemma}%[Appendix of \cite{bp}]
%\label{lemma:limit}
%Let $(t_n: n <\omega)$ be a sequence of trees that converges to $t^*$ and let $(\sigma_n: n \in \omega)$ be a sequence of type trees that converges to $\sigma^*$. If $\sigma_n$ is locally consistent with $t_n$ for every n, then $\sigma^*$ is locally consistent with $t^*$.
%\end{lemma}

\subsubsection*{Characterizing countable Wadge degrees.}
\label{sec:the main result}
Everything now is ready to prove the main result of this paper.
\begin{theorem}\label{theorem:main}
Let $L$ be a regular tree language given by a non-deterministic tree automaton $\A$. The following conditions are equivalent:
\begin{enumerate}
%\item The following identity is not satisfied: $(u_2w_2^\omega v)^\omega u_1w_1^\infty = (u_2w_2^\omega v)^\infty$ if $(u_1, u_2) \in V_L$ and $(w_1, w_2) \in V_L$
%\item $L$ has unbounded limit alternation
\item The strategy graph $G_L$ is recursive.
\item $d_W(L) \geq \omega_1$
\end{enumerate}
In particular, since the graph $G_L$ is computable from the automaton $\A$, it is decidable whether the language accepted by $\A$ is of Wadge degree bigger or eqal than $\omega_1$.  
\end{theorem}

\begin{proof}
%The implication $(1) \Rightarrow (2)$ is proved in \cite{bp}.  
We start by showing that $(1) \Rightarrow (2)$. Assume the strategy graph is recursive. This means that there exists a strongly connected component that contains two nodes $(v, h)$ and $(v', h')$ with $h \neq h'$. 
Thanks to Proposition \ref{proposition:path to edge}, if there exists a path between $(v, h)$ and $(v', h')$, there is also an edge between $(v, h)$ and $(v', h')$. 
Moreover, for vertices $(v_1,h_1),(v_2,h_2),\dots$, if for every $i=1,2,\dots$ there is an edge from $(v_i, h_i)$ to $(v_{i+1}, h_{i+1})$, this means that Alternator has a winning strategy in $\mathcal{H}(h_1,h_2,\dots)$. So, take $(v_i, h_i)= (v,h)$ for $i$ even, and $(v_i, h_i)= (v',h')$ for $i$ odd. We have that Alternator has a winning strategy in $\mathcal{H}^\varepsilon_\infty(h, h')$. By Proposition \ref{prop:tree_to_types}
Alternator has a winning strategy in  $\mathcal{H}^\varepsilon_\infty(L, L^\complement)$.


Now we turn to the implication  $(2) \Rightarrow (1)$. 
By Propositions \ref{prop:infinity}  and \ref{prop:locality}, it is enough to verify that 
 if Alternator has a winning strategy in $\mathcal{H}^\varepsilon_\infty(L, L^\complement)$ then there is a set of locally optimal finite strategy trees $\mathfrak{S}$ with both root and limit unbounded alternation. 
 
Assume Alternator has a winning strategy $f$ in  $\mathcal{H}^\varepsilon_\infty(L, L^\complement)$. From Proposition \ref{prop:types} there is a strategy tree $\mathfrak{s}^\infty = (t,\sigma_1,\dots) $ with infinite root alternation. 
 By Lemma \ref{lemma:locallyoptimal} we can assume that $\mathfrak{s}^\infty$ is  locally optimal. Let us define 
\[ \mathfrak{S}= \{ (t,\sigma_1,\dots,\sigma_k): k=1,2,\dots\}. \]
Note that each element of $\mathfrak{S}$ is locally optimal.
Now, assume limit alternation of $\mathfrak{S}$ is bounded.
This means that there is a strategy tree $\mathfrak{s}=(t, \sigma_1, \dots, \sigma_j)$ $ \mathfrak{S}$, and there are $k, k', i, \ell, $ with $k < k' \leq j$ such that
 there is a set $\{n_1, \dots, n_\ell\}$ of nodes of $t$ included in a prefix of depth $i$ satisfying the following properties:
\begin{itemize}
%\item there is a node $n^* \in \{n_1, \dots, n_\ell\}$ with two children $m_1, m_2 \in \dom(t)|_i \setminus \{n_1, \dots, n_\ell\}$
\item $\sigma_k(n_i)\neq \sigma_{k'}(n_i)$ for $i=1, \dots, \ell$,
\item $\sigma_k(n)= \sigma_{k'}(n)$, for every $n \in \dom(t)|_i \setminus \{n_1, \dots, n_\ell\}$.
\end{itemize}
Assume $\mathfrak{s}=(t, \sigma_1, \dots, \sigma_j)$, and let us consider the game $\mathcal{H}( \sigma_1(\varepsilon), \dots, \sigma_j(\varepsilon))$ where at first Alternator plays $t$ and then Constrainer uses the strategy given by cutting always at level $i+1$. By Lemma \ref{lemma:short_strategy}, the trees played at round $k$ and $k'$ using her winning strategy are the same, say $t'$. But by locally consistency there is a node $n$ such that the subtree $t'.n$ of $t'$ with root in $n$ has both type $\sigma_k(n)$ and $\sigma_{k'}(n)$, with $\sigma_k(n) \neq \sigma_{k'}(n)$. Contradiction.
  \end{proof}
