\section{A characterization of regular languages  below $\omega_1$}

%We recall that, given a language $L$, the set $H_L$ is the set of tree types, and the set $V_L$ of context types of $L$. Both are finite when $L$ is regular. 
Following \cite{bp}, for a given regular language of trees $L$ and types $h_i\in H_L$ ($i=1,2,\dots$) we define games on types $\mathcal{X}(h_1, \dots, h_n)$ and $\mathcal{X}(h_1, h_2, \dots)$. The Constrainer plays as before and the task of the Alternator is to play in the $i$--th round a tree of type $h_i$, that is an element of $\alpha_L^{-1}(h_i)$. 
% deleted, because there is seemingly no place where we play contexts. 
% Similarly we define games on types of contexts $\mathcal{X}(v_1, \dots, v_n)$ and $\mathcal{X}(v_1, \dots, v_n, \dots)$. The Constrainer plays as before and the task of the Alternator is to play in the $i$--th round a tree of type $v_i$
Thus we define
\[ \mathcal{H}_L:= \{(h_1, \dots, h_n) \mid n \in \omega\ \mbox{and Alternator wins}\ \mathcal{X}(h_1, \dots, h_n)\}, \]
\[ \mathcal{H}^\infty_L:= \{(h_1, \dots,) \mid n \in \omega\ \mbox{and Alternator wins}\ \mathcal{X}(h_1, \dots )\}.\]
%and $\mathcal{H}^\infty_L 
%$\mathcal{V}_L:= \{(v_1, \dots, v_n) \mid n \in \omega 
%$
%and Alternator wins $\mathcal{X}(v_1, \dots, v_n)\}$.
%Sets 
%$\mathcal{H}^\infty_L$ and $\mathcal{V}^\infty_L$ are defined analogously.
%For a multicontext $c$, we can also define the game $\mathcal{X}^c(h_1, \dots, h_n, \dots)$
%the same way as $\mathcal{X}(h_1, \dots, h_n, \dots)$ with the added restriction that the first tree played by alternator must extend the multicontext $c$. 


A type tree for $L$ is a tree over the finite alphabet $\mathcal{H}_L$. For a given tree $t$, there is a type tree $\sigma_t$ induced by $t$ such that for every node $w \in \dom(\sigma_t)$, $\sigma_t(w)$ is the type of the tree $t.w$.

Let $\sigma$ be a type tree, and $t$ a tree. We say that a type tree $\sigma$ is locally consistent with a tree $t$ if $\dom(\sigma)=\dom(t)$ and for every node $w \in \dom(t)$ such that $t(w)=a$, 
\begin{itemize}
\item if $w$ is a leaf, then $\sigma(w)$ is the type of $a$,
\item if $w$ has two children $m_\ell$ and $m_r$, then $\sigma(w)$ is the type obtained by applying $a$ to the pair $(\sigma(m_\ell), \sigma(m_r))$.
\end{itemize}

A finite strategy tree is a tuple
\[ \mathfrak{s}=(t, \sigma_1, \dots, \sigma_k) \] where
\begin{itemize}
\item $t$ is a tree, the support of the strategy
\item $\sigma_t=\sigma_1$, and
\item $\sigma_\ell$ is locally consistent with $t$, $\ell \leq k$,
\item for each $w \in \dom(t)$, $(\sigma_1(w), \dots, \sigma_k(w)) \in \mathcal{H}_L$.
\end{itemize}
An  infinite strategy tree is defined analogously.

Given a prefix $p$, we can associate a multi context $c_p$ given by adding two holes to every leaf of $p$. For a context $c$, the set $Holes(c)$ is the set of all holes of $c$. The Lemma below follows immediately from the definition of a strategy tree.
\begin{lemma}\label{lemma:short_strategy}
Let $\mathfrak{s}=(t, \sigma_1, \dots, \sigma_k)$ be a any strategy tree. Given the  game $\mathcal{X}(\sigma_1(\varepsilon), \dots, \sigma_k(\varepsilon))$ and a strategy for Constrainer given by always cutting at level $i$, Alternator wins  by playing as follows:
\begin{itemize}
\item at first, Alternator plays $t$, then
\item for each hole $w$ at level $i$ of the multi context given by Constrainer's move, Alternator plugs in the tree given by her winning strategy $\mathcal{X}(\sigma_1(w), \dots, \sigma_k(w))$.
\end{itemize}
In particular, if from a certain $j<k$ on $\sigma_\ell(w)=\sigma_{\ell+1}(w)$, $j\leq \ell < k$, then for each round $j< \ell < k$ she always plugs in the same tree of type $\sigma_j(w)$ chosen at round $j$.
\end{lemma}
Let $\mathfrak{s}$ be a finite strategy tree. The root alternation of $\mathfrak{s}$ is the alternation of the root sequence, while the limit of alternation of $\mathfrak{s}$ is the maximal number $k$ such that infinitely many subtrees of $\mathfrak{s}$ have root alternation $k$. 

We say that $L$ has bounded root alternation if the set $\mathfrak{S}_L$ of finite strategy trees has bounded root alternation. Analogously for limit alternation.
The following proposition is a straightforward generalization of an analogous proposition (5.2) in \cite{bp}:
\begin{proposition}\label{prop:tree_to_types}
 For a regular language $L$ of infinite trees and any prefix $p$, the following conditions are equivalent.
 \begin{itemize}
\item Alternator wins the game $\mathcal{X}^p_\infty(L, L^\complement)$, 
 \item  There are tree types $h, g\in H_L$, such that $h\neq g$ and Alternator wins $\mathcal{X}^p_\infty(h, g)$.
 \end{itemize}
\end{proposition}
The following Lemma will turn out to be useful:
\begin{lemma}[Appendix of \cite{bp}]\label{lemma:limit}
Let $(t_n: n <\omega)$ be a sequence of trees that converges to $t^*$ and let $(\sigma_n: n \in \omega)$ be a sequence of type trees that converges to $\sigma^*$. If $\sigma_n$ is locally consistent with $t_n$ for every n, then $\sigma^*$ is locally consistent with $t^*$.
\end{lemma}
Everything now is ready to prove the main result of this note.
\begin{theorem}
Let $L$ be a regular tree language. The following conditions are equivalent:
\begin{enumerate}
%\item The following identity is not satisfied: $(u_2w_2^\omega v)^\omega u_1w_1^\infty = (u_2w_2^\omega v)^\infty$ if $(u_1, u_2) \in V_L$ and $(w_1, w_2) \in V_L$
\item $L$ has unbounded limit alternation
\item The strategy graph $G_L$ is recursive
\item $d_W(L) \geq \omega_1$
\end{enumerate}
\end{theorem}

\begin{proof}
The implication $(1) \Rightarrow (2)$ is proved in \cite{bp}.  
We show that $(2) \Rightarrow (3)$. Assume the strategy graph is recursive. This means that there exists a strongly connected component that contains two nodes $(v, h)$ and $(v', h')$ with $h \neq h'$. 
Now, if there exists a path between $(v, h)$ and $(v', h')$, there is also an edge between $(v, h)$ and $(v', h')$. 
Moreover, for every sequence $((v_i,h_i) : i \in \alpha)$, for $\alpha \leq \omega$, if there is an edge from $(v_i, h_i)$ to $(v_{i+1}, h_{i+1})$, this means that Alternator has a winning strategy in $\mathcal{X}(h_i: i \in \alpha)$. So, take $(v_i, h_i)= (v,h)$ for $i$ even, and $(v_i, h_i)= (v',h')$ for $i$ odd. We have that Alternator has a winning strategy in $\mathcal{X}^\varepsilon_\infty(h, h')$. By Proposition \ref{prop:tree_to_types}
Alternator has a winning strategy in  $\mathcal{X}^\varepsilon_\infty(L, L^\complement)$.


Now we turn to the implication  $(3) \Rightarrow (1)$. 
By proposition \ref{prop:infinity} it is enough to verify that 
 if Alternator has a winning strategy in $\mathcal{X}^\varepsilon_\infty(L, L^\complement)$ then $L$ has unbounded limit alternation.
 
Assume Alternator has a winning strategy $f$ in  $\mathcal{X}^\varepsilon_\infty(L, L^\complement)$. We are going to verify that there is a infinite strategy tree $\mathfrak{s}^\infty$ with infinite root alternation. The strategy tree is constructed as follows. First of all, $f$ can be represented as a tree $t_f$: 
\begin{itemize}
\item the root is labelled by $\varepsilon$, and its unique child is labelled by Alternator's move given by $f(\varepsilon)$,
\item if a node $n$ at odd depth (and thus representing Alternator's move) is labelled by a tree $t_n$, then for every prefix $p$ of $t_n$, there is a unique child $n$ labelled by $p$,
\item if a node $n$ at even depth (and thus representing Spoiler's move) is labelled by a prefix $p$ of the tree $t_f(m)$, which is the label of the unique ancestor $m$ of $n$, then there is a unique child $n'$ of $n$ labelled by $f(t_f(m_1)\dots t_f(m_k))$, where $(m_1\dots m_k)$ is the path from the root of the tree to $m$.
\end{itemize}
Thus from now on we identify $f$ and $t_f$.

Given a node $n$ of $f$ labelled by a prefix $p$, and $\alpha < \omega$,  by $(f.n)|_k$, we denote the induced  winning strategy  for Alternator in  $\mathcal{X}^p_\alpha(L, L^\complement)$, if $n$ is at depth $2i$ with $i$ even, and in  $\mathcal{X}^p_\alpha(L^\complement, L)$, if $n$ is at depth $2i$ with $i$ odd.

Now, if we prove the following claim we are done:
\begin{claim}\label{claim:strategy}
For every node $n$ of $f$ labelled by a prefix $p$, 
there is a sequence of strategy trees $\mathfrak{S}_n=(\mathfrak{s}^n_i: i < \omega)$ such that for each $n$
\begin{enumerate}
\item $\mathfrak{s}^n_\ell=(t, \sigma_1, \dots, \sigma_\ell)$, with $\sigma_{2k+1}$ the type of $L$ and $\sigma_{2k}$ the type of  $L^\complement$ if $n$ is at depth $2i$ with $i$ even, else dually,
\item $\mathfrak{s}_{n+1}$ extends $\mathfrak{s}_n$,
\end{enumerate}
\end{claim}
Indeed, given the claim above, from point 1 we have that
for each node $n$ labelled by a prefix $p$, and each $k < \omega$, $\mathfrak{s}^n_k$ has root alternation $k$ and defines a winning strategy for Alternator in $\mathcal{X}^p_k(L, L^\complement)$ if $n$ is at depth $2i$ with $i$ even, in $\mathcal{X}^p_k(L^\complement, L)$ else.
From point 2, it is thus enough to take $\mathfrak{s}^\infty$ has the limit of the sequence $(\mathfrak{s}^\varepsilon_k: k < \omega)$, and $\mathfrak{S}=\bigcup_{k \in \omega} \mathfrak{s}^\varepsilon_k$. 
Now, assume limit alternation of $\mathfrak{S}$ is bounded.
This means that there is a strategy tree $\mathfrak{s} \in \mathfrak{S}$, and there are $k, k', i, \ell, $ with $k < k'$ such that
 there is a set $\{n_1, \dots, n_\ell\} \subset \dom(t)|_i$, where $\dom(t)|_i$ denotes the set of nodes of $t$ of depth at most $i$, and
\begin{itemize}
%\item there is a node $n^* \in \{n_1, \dots, n_\ell\}$ with two children $m_1, m_2 \in \dom(t)|_i \setminus \{n_1, \dots, n_\ell\}$
\item $\sigma_k(n_i)\neq \sigma_{k'}(n_i)$ for $i=1, \dots, \ell$,
\item $\sigma_k(n)= \sigma_{k'}(n)$, for every $n \in \dom(t)|_i \setminus \{n_1, \dots, n_\ell\}$.
\end{itemize}
Assume $\mathfrak{s}=(t, \sigma_1, \dots, \sigma_j)$, and let us consider the game $\mathcal{X}( \sigma_1(\varepsilon), \dots, \sigma_j(\varepsilon))$ where at first Alternator plays $t$ and then Constrainer uses the strategy given by cutting always at level $i+1$. By Lemma \ref{lemma:short_strategy}, the trees played at round $k$ and $k'$ using her winning strategy are the same, say $t'$. But by locally consistency there is a node $n$ such that the subtree $t'.n$ of $t'$ with root in $n$ has both type $\sigma_k(n)$ and $\sigma_{k'}(n)$, with $\sigma_k(n) \neq \sigma_{k'}(n)$. Contradiction.



We prove the claim \ref{claim:strategy} by verifying by induction that the sequence $(\mathfrak{s}^n_i: i < k)$ can be extended to a sequence $(\mathfrak{s}^n_i: i < k+1)$ satisfying the properties 1 and 2 of the claim. 



In order to do so, we reason as follows. First of all, given a sequence of types $(\sigma_i : i \in \omega)$, by compactness there is a converging subsequent whose limit is  $\sigma^*$. We assume such a subsequence given: every time we are going to choose the limit of a converging  subsequence of the sequence $(\sigma_i : i \in \omega)$, we are going to take $\sigma^*$.

Now, for the initial step, it is enough to take for each node $n$ and prefix label $p$, $\mathfrak{s}^n_1=(t, \sigma_1)$, where $t$ is given by applying $f$ to the considered position, and $\sigma_1$ is the unique tree type induced by $t$. 

We now describe the general construction for $k>1$. Fix any node $n$ with label a prefix $p(n)$. Assume the answer given by $f$ is $t$. For every prefix $p$ of $t$, we have a strategy tree $\mathfrak{s}^m_{k-1}=(t_p, \sigma_{p2}, \dots, \sigma_{pk})$, where $m$ is the node corresponding to $p$ in $f$. 
By compactness, there is an infinite subsequence of prefixes $(p_i: i < \omega)$ whose limit is $t$ and such that all of the sequences
$(t_{p_i}: i <\omega ), (\sigma_{p_i2}: i <\omega ), \dots,  (\sigma_{p_ik}: i <\omega )$ are convergent. Let the limits of these sequences be $t, \sigma^*_2, \dots, \sigma^*_k$.
For each $p$, the type trees $(\sigma_{p2}, \dots, \sigma_{pk})$
are locally consistent with $t_p$. Therefore, by Lemma \ref{lemma:limit} it follows that the limits $\sigma^*_2, \dots, \sigma^*_k$ are locally consistent with $t$. Finally, define $\sigma^*_1$ to be the unique type tree induced by  $t$ (globally consistent with it). We have just proved that
$(t, \sigma^*_1, \dots, \sigma^*_k)$
is a strategy tree. Because root values are preserved under limits, the root value
of this strategy tree is of the desired kind. By induction hypothesis, such a construction has the property that $\mathfrak{s}^m_{k+1}$ extends $\mathfrak{s}^m_{k}$. This concludes the proof of the claim.
  \end{proof}
