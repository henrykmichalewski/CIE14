% !TEX root = FacMik.tex
\section{Topological complexity and games}\label{section:games}
\subsubsection*{Topological Games.}
Let $L$ and $M$ be two languages. The {\em Wadge game}
$\mathcal{W}(L, M)$ is an infinite two-player game between Player I and Player II. It is defined as follows. During a play Player I constructs a tree $t$ and Player II a tree $t'$. At the first round Player I plays a root of $t$ and Player II plays a root of $t'$, and at each consecutive round both players add a level
to their corresponding tree (thus either Player adds some child to a leaf or Player signalizes that the node will be also a leaf of the final resulting tree of the play by not adding any children to it).
Player I plays first and Player II is allowed to
skip her turn but not forever.  Player II wins the game iff $t \in L
\Leftrightarrow t' \in M$.  
The game was designed precisely in order to obtain a characterisation of continuous reducibility.
\begin{lemma}[\cite{wadge}]\label{lemma:wadge}
Let $L, M$ be two languages. Then  $L \leq_W M$ iff Player II has a winning strategy in the game $\mathcal{W}(L, M)$.
\end{lemma}
From Borel determinacy (\cite{martin}),
if both $L$ and $M$ are Borel, then $\mathcal{W}(L, M)$ is determined.
As a consequence, a variant of Martin-Monk's result shows that $<_W$ is
well-founded. The \emph{Wadge degree} for sets of finite Borel rank is
inductively defined as follows. First, we remark that since every self-dual set $A$ is Wadge equivalent to the disjoint union of non self-dual set $B$ and its complement, it is enough to start associating a Wadge degree only to non self-dual sets and say that the Wadge degree of $A$ equals the Wadge degree of $B$. Hence, the mapping associating to each non self-dual set its Wadge degree is given by:
\begin{itemize}
\item $d_W(\emptyset)=d_W(\emptyset^\complement)=1$,
\item $d_W(L)=\sup\{d_W(K)+1\colon K \text{ non self-dual}, K <_WL\}$ for $L>_W\emptyset$.
\end{itemize}
The
height of the Wadge hierarchy of all tree languages (recognizable or
not) of finite Borel rank is $\,{}^{\omega_1} \epsilon_0$ (see \cite{dup1}).

For each degree there are exactly three equivalence
classes with the same degree, represented by $L$, $L^\complement$ and
$L^\pm$ --- the disjoint union of $L$ and $L^\complement$. %  = \{t \bigm | t(\epsilon) = c, t.0\in L \} \cup  \{t \bigm |
%t(\epsilon) \neq c, t.0\notin L \}$ for some non self-dual set $L$ and
%$c\in A$. 
Clearly $L, L^\complement <_W L^\pm$ and
$L^\pm$ is self-dual.

In \cite{dup1}, J. Duparc showed that for non self-dual sets, it is possible to determine its sign, $+$
or $-$, which specifies precisely the $\equiv_W$-class. 
%For a set $T$ whose Wadge degree has countable cofinality, the sign
%is $+$ if $T$ is Wadge equivalent to the set of trees over $A
%\cup \{c\}$, $c\notin A$, which have no $c$ on the leftmost
%branch, or the first $c$ is in the node $0^i$ and $t.0^i \in T$. The
%sign is $-$ if $T$ is equivalent to the complement of this set. 
For instance, $\emptyset$ and complete open sets have sign $-$, while the whole space and complete closed  sets have sign $+$.
%For sets of cofinality $\omega_1$, the definition is more complicated, but
%${\bf \Sigma}^0_n$-complete sets have sign $-$, and ${\bf
  %\Pi}^0_n$-complete sets have sign $+$. 
  All self-dual sets by
definition have sign $\pm$. Thus an ordinal $\alpha <\,
{}^{\omega_1}\epsilon_0$ and a sign $\epsilon \in \{+,-,\pm\}$,
determine a $\equiv_W$-class, denoted  $[\alpha]^\epsilon$. % For each $n>1$, ${\bf \Sigma}^0_n$-complete has signed Wadge degree $[\exp^{n-1}(1)]^-$, and  ${\bf \Pi}^0_n$-complete has thus signed degree $[\exp^{n-1}(1)]^+$, where $\exp(\alpha) = \omega_1^\alpha$ and $\exp^{k+1}(\alpha) = \omega_1^{\exp^k{\alpha}}$.
In the same paper, Duparc defined the exact set-theoretical counterpart of the ordinal operations on signed Wadge degree of sum, multiplication by a countable ordinal, and (quasi) exponentiation of base $\omega_1$. In this way he was able to generate from the empty set and its complement canonical complete sets for each signed Wadge degree $[\alpha]^\epsilon$, with $\alpha <\,
{}^{\omega_1}\epsilon_0$. 
%Because of lack of space, we do not present such constructions, but we give some intuitions on  some of those operations and sets from the perspective of Wadge games (cf. for instance the description in \cite{dup3}). 
From now on with $[\alpha]^\epsilon$ we also denote the canonical sets of Wadge degree generated with Duparc's operations. We present more details of Duparc's construction in the Appendix.
%%%%%%%%%
%CUTTING
%%%%%%%%%
\subsubsection*{Cutting games.}
%\label{cuttinggames}
%\subsection{Delayed cutting games}
Below we define a family of two-player games of perfect information, called \emph{cutting games}. These games were
introduced in \cite{bp}. For the argument in \cite{bp} the most important was the finite version of the game. In the present paper 
we will consider both infinite and finite versions of this game. % and some variations of the finite games from \cite{bp}. These variations we call \emph{delayed cutting games}. 

Let $L_i$ ($i=1,2,\dots$) are languages over the alphabet $A$, and $p$ is a prefix over the alphabet $A$. 
The \emph{simple cutting game} of length $k$, denoted $\mathcal{H}^p_k(L_1,\ldots,L_k)$ is played by two players, Constrainer and Alternator. 
For each $i \in \{1, \dots, k \}$ the $i$-th round of the game is played as follows:
\begin{itemize}
\item Alternator chooses a tree $t_i \in L_i$ extending the prefix chosen in the previous round by the Constrainer; in the first round of the game Alternator must choose an extension of the given prefix $p$,
\item Constrainer chooses a prefix of the tree $t_i$.
\item If Alternator cannot move, she loses, if she manages to survive $k$ rounds, then she wins.
\end{itemize}
The \emph{infinite cutting game}, denoted by $\mathcal{H}^p_\infty(L_1,\dots)$, is played just like a simple game but without the restriction to a fixed given number of rounds. Alternator wins iff she can make infinitely many moves.

Let $X$ be a language over the alphabet $A$.  The \emph{$X$-delayed cutting game}, denoted by 
$\mathcal{H}^{X}_\omega(L_1,\dots)$ is similar to a simple cutting game, except that a mini game is played to determine the prefix $p$ and the length $k$ of the match. 
The mini game goes as follows. Firstly, Alternator chooses a tree $t \in X$. Then Constrainer chooses a prefix $p$ of $t$ and a finite ordinal $k$. Finally the two players starts to play the simple cutting game $\mathcal{H}^p_k(L_1,\dots,L_k)$. 

When $L_{2i}=L$ and $L_{2i+1}=L^\complement$, then we simply write $\mathcal{H}^p_k(L, L^\complement), \mathcal{H}^p_\infty(L, L^\complement)$ and  $\mathcal{H}^{X}_\omega(L, L^\complement)$.
It was verified in \cite{bp} that a given language $M$ has a Wadge degree less than $\omega$ iff Constrainer has a winning strategy in $\mathcal{H}^\varepsilon_k(M, M^\complement)$, for all but finitely many $k<\omega$.
In \cite{bp} it was also remarked that the language $L$ described in Proposition \ref{rem:example}, even  if it is such that Alternator has a winning strategy in every corresponding finite cutting game, she looses the infinite one.
In the next two propositions we establish link between delayed cutting games and infinite Wadge degrees on the one hand, and infinite simple cutting games and uncountable Wadge degrees on the other hand.
% we provide the expected explanation of this fact.
\begin{proposition}\label{prop:omega} Let $L$ be a tree language, $[\omega]^+ \leq_W L$ iff Alternator has a winning strategy in $\mathcal{H}^{L}_\omega(L^\complement, L)$. 
\end{proposition}
\begin{proposition}\label{prop:infinity} Let $L$ be a tree language. For every prefix $p$,
$d_W([p]^{-1}L) \geq \omega_1$ iff Alternator has a winning strategy in $\mathcal{H}^p_\infty(L, L^\complement)$. 
\end{proposition}
The proofs of the previous Propositions \ref{prop:omega} and \ref{prop:infinity} are in the Appendix. 

