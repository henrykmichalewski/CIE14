% !TEX root = FacMik.tex

\section{Conclusion}
The algorithm provided in \cite{bp} decides whether a given non-deterministic automaton  accepts a language which is a Boolean combination of open sets or equivalently is of a  Wadge degree smaller than $\omega$. 
%In this paper using 
By the same approach we showed an algorithm which decides whether a given non-deterministic automaton accepts a language in ${\bf \Delta^0_2}$ or equivalently, a language of a Wadge degree smaller than $\omega_1$. 
We propose for further research the following three generalizations of the result presented in this paper:

\noindent
{\bf 1}. % We do not know how to generalize the presented method to Wadge degrees between $\omega$ and $\omega_1$ despite the fact that certain aspects of the proof remain intact. Namely, 
For a given $n=1,2,\dots$ there are natural topological games which characterize languages of Wadge degrees smaller than $\omega^n$. Moreover, there are known examples of regular languages of  degree $\omega^n$. It would be a desirable and perhaps more involved extension of results in \cite{bp} if for a given $n$ one can provide an algorithm deciding whether a given non-deterministic automaton accepts a language of degree smaller than $\omega^n$. 

\noindent
{\bf 2.} In the absence of examples of regular languages between Wadge degree $\omega^\omega$ and Wadge degree $\omega_1$, one could reasonably expect, that the decidability result in the present paper should show that indeed any regular language of countable Wadge degree is of Wadge degree smaller than $\omega^\omega$. However, this question still remains open.  

\noindent
{\bf 3}. Regarding higher Borel classes, in particular regular languages which are Boolean combinations of ${\bf \Sigma^0_2}$ sets, the following extension of the method in \cite{bp} seems to be plausible. The cutting game is based around restrictions of moves by prefixes, that is 
%by clopen sets, or equivalently 
by languages in ${\bf \Delta^0_1}$. It's  topological counterpart on the next Borel level is a game, where the Constrainer is allowed to play constraints which are regular languages in ${\bf \Delta^0_2}$. This leads to natural topological characterization similar to the results in Section \ref{section:games}, but the algebraic counterpart of this generalized cutting game is not yet fully understood.
