\section{Conclusion}
The algorithm provided in \cite{bp} decides whether a given non-deterministic automaton $\A$ accepts a language which is Boolean combination of open sets or equivalently is of a  Wadge degree smaller than $\omega$. In this paper using the same approach we showed an algorithm which decides whether a given non-deterministic automaton $\A$ accepts a language in ${\bf \Delta^0_2}$ or equivalently, a language of a Wadge degree smaller than $\omega_1$. 

We do not know how to generalize this method to intermediate steps between $\omega$ and $\omega_1$ despite the fact that certain aspects of the proof remain intact. Namely, there are natural topological games which characterize languages of degrees smaller than $\omega^n$. Moreover, for every $n=1,2,\dots$ there are known examples of regular languages of trees on the level $\omega^n$ of the Wadge hierarchy. It would be a desirable and perhaps more involved extension of results in \cite{bp} if for a given $n$ one can provide an algorithm deciding whether a given non-deterministic automaton $\A$ accepts a language of degree smaller than $\omega^n$. Similarly, in the absence of examples of regular languages with Wadge degree smaller than $\omega_1$ and bigger or equal $\omega^\omega$ one could reasonably expect, that the decidability result in this paper should show that indeed any regular language of Wadge degree smaller than $\omega_1$ is of Wadge degree smaller than $\omega^\omega$. However, this question still remains open.  

Regarding higher Borel classes, in particular regular languages which are Boolean combinations of ${\bf \Sigma^0_2}$ sets, the following extension of the method in \cite{bp} seems to be plausible. The cutting game is based around restrictions of moves by prefixes, that is by clopen sets, or equivalently by languages in ${\bf \Delta^0_1}$. It's  topological counterpart on the next Borel level is a game, where the Constrainer is allowed to play constraints which are regular languages in ${\bf \Delta^0_2}$. This leads to natural topological characterization similar to the results in the Section \ref{cuttinggames}, but the algebraic counterpart of this generalized cutting game is not yet fully understood.
