\section{Introduction}
In \cite{niwalu} there was given an algorithm which for a \emph{deterministic} parity tree automaton $\A$ decides whether the language $L(\A)$ is Borel. This was further extended to a finer classification in \cite{niwalu2} and finally to a full Wadge classification in \cite{murlak}. The algorithms look for a pattern in the graph of the automaton and decide the Borel and Wadge classes upon finding of these special patterns. 

For a given non--deterministic parity tree automaton $\A$ in \cite{bp} was provided an algorithm which decides
whether $L(\A)$ is a Boolean combination of open sets. For other Borel classes there was no known algorithm. This paper provides a relatively simple extension 
of the result in \cite{bp} to the class of ${\bf \Delta^0_2} = {\bf\Sigma^0_2}\cap{\bf\Pi^0_2}$ sets, that is the sets which are simultaneously presentable as countable unions of closed sets and countable intersections of open sets. The proofs in \cite{bp} are based on an analysis of algebraic structure computable from $\A$ and the main result states that the language $L(\A)$ is a Boolean combination of open sets if and only if a certain finite number algebraic requiremens hold. Since the class ${\bf \Delta^0_2}$ is bigger, in order to characterize it the set of algebraic requirements must be relaxed, that is made smaller. In the present paper we show that indeed it is the case. Our proofs closely follow the proofs from \cite{bp} with some necessary adjustments. In particular the crucial concept of the topological cutting game introduced in \cite{bp} is considered in the paper not only in the finite, but also in the infinite case. 

The approach presented in \cite{bp} and in this paper in a certain sense is a reminiscent of the approach applied to deterministic automata in \cite{niwalu,niwalu2,murlak}. Namely, the algebraic structure induces a graph where edges reflects the algebraic properties. As in the deterministic case it is possible to decide Borel and Wadge classes analyzing patterns in the graph of the automaton, in this paper we are looking for patterns in the algebraic graph. 

Finally let us mention two results which link a given non--deterministic automaton with its set-theoretical complexity. In \cite{rabin} it was proved, that if $L$ and its complement are accepted by a non-deterministic Buchi tree automaton, then the language is weakly definable, in particular it is Borel. The Rabin method together with decidability results about the cost functions lead recently to a new algorithm in \cite{cklvb}, which given a non-deterministic Buchi tree automaton $\A$, decides whether the language is weakly definable. 
