\section{Preliminaries}
\subsection{Trees and contexts}
Given a finite alphabet $A$, a \emph{binary tree} over $A$ (from now on simply a tree) is a partial function $t: \{0,1\}^* \to A$ such that its domain $\dom(t)$ is prefix closed. A node of a tree $t$ is just an element $n \in \dom(t)$. A left child of a node $n$ of $t$ is the node $n0$, while its right child is $n1$. A leaf of a tree is a node without children. A tree $t$ over $A$ is infinite if it has no leaves (and thence it is a total function). We denote by $T_A$ the class of all  trees over $A$ and $T^\infty_A$ of all infinite trees over $A$. A set of trees over $A$ is called a tree language, or simply a language.
A \emph{multi-context} over $A$ is a tree $c$ over $A \cup \{\star\}$, where 
\begin{itemize} 
\item $\star \notin A$, and 
\item $\star$ only labels some leaves of $c$ (at least one). 
\end{itemize}
A leaf of $c$ labelled by $\star$ is called a \emph{port}. Notice that a multi-context may have infinitely many ports. For a multi-context $c$ and a function $\eta$ mapping each port of $c$ to a tree $t$ over $A$, by $c[\eta]$ we denote the tree given by inserting in to every port $x$ a tree $\eta(x)$. The class generated by $c$ and all possible mappings $\eta$ is denoted by $c[T_A]$, and by $c[T^\infty_A]$ we denote the class generated by $c$ and all possible mappings $\eta$ with range contained in $T_A^\infty$. 

A finite multi-context is called a \emph{prefix}.
A multi-context with only one port is called a \emph{context}. %For an alphabet $A$, the class of all contexts over $A$ is denoted by $V_A$.
%% main text

\subsection{Topology}
For a finite alphabet $A$, we equip the class $T^\infty_A$ of all infinite trees over $A$ with the prefix topology. That is the basic open sets are sets of the form $p[T^\infty_A]$, for a prefix $p$ over $A$, and thus the open sets are of the form $\bigcup_{p \in P}p[T^\infty_A]$ for some set $P$ of prefixes. 

The class of Borel tree languages of $T^\infty_A$ is the
closure of the class of open sets of $T^\infty_A$ by countable unions and
complementation. Given $T^\infty_A$, the initial finite levels of the
Borel hierarchy are defined as follows with ${\bf\Sigma}^0_0(T^\infty_A)= \{\emptyset\}$ and
${\bf\Pi}^0_0(T^\infty_A)=\{T^\infty_A \}$.
\begin{itemize}
\item ${\bf\Sigma}^0_1(T^\infty_A)$ is the class of open subsets of $T^\infty_A$, 
\item ${\bf\Pi}^0_n(T^\infty_A)$ contains complements of sets from ${\bf\Sigma}^0_n(T^\infty_A)$, 
\item ${\bf\Sigma}^0_{n+1}(T^\infty_A)$ contains countable unions of sets from ${\bf\Pi}^0_n(T^\infty_A)$. 
\end{itemize}

A much finer measure of the topological complexity of tree languages is the \emph{Wadge degree} (see \cite[Chapter 21.E]{kechris}).
If $L \subseteq T^\infty_A$ and $M\subseteq T^\infty_A$, we say that $L$ is \emph{continuously (or Wadge)
reducible} to $M$, if there exists a continuous function $f$ such that $L=
f^{-1}(M)$. We write $L \leq_W M$ iff $L$ is continuously reducible to $M$.
This ordering is called the {\em Wadge ordering}. If $L \leq_W M$ and $M \leq_W L$, then we write $L
\equiv_W M$. If $L \leq_W M$ but not $M \leq_W L$, then we write $L<_W
M$. The Wadge hierarchy is the partial order induced by $<_W$ on the
equivalence classes given by $\equiv_W$. A language  $L$ is called {\em self dual} if it is equivalent 
to its complement, otherwise it is called {\em non self dual}.


Given a certain family of sets $\mathcal{C}$, we say that $M$ is $\mathcal{C}$-hard if $L
\leq_W M$ for every $L \in \mathcal{C}$.   

\subsection{Algebra}
The Wadge hierarchy on the regular languages of infinite words is well understood thanks to a classification result by K. Wagner (\cite{wagner}). A natural algebraic interpration of Wagner's result can be found in \cite[Theorem 6.2]{pinperrin}. In the case of languages of infinite trees, the algebraic theory is not yet fully developed. As a general reference may serve papers \cite{blumensath}, \cite{bp}, \cite{bojtrees}, \cite{bojidziaszek}. For details of the approach applied in this paper refer to \cite[Section 3]{bp}. 

As in \cite{bp} family of all trees $T_A$ is divided into finitely many Myhill-Nerode equivalence classes $H_L$. Similarly, there are finitely many equivalence classes $V_L$ of contexts. Starting from an automaton accepting language $L$, one can compute families $H_L$ and $V_L$. The equivalence class of a tree $t$ or a context $v$ is denoted $\alpha_L(t)$, $\alpha_L(v)$, respectively. For a given tree $t$ and contexts $v_1,v_2$ multiplication of contexts and trees $v_1t$, $v_1v_2$ naturally induces multiplication between elements of $H_L$ and $V_L$. Similarly, for a given context $v$ the operation of infinite power $vv\ldots$ induces a mapping from $V_L$ to $H_L$. 


