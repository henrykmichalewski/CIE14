% This is LLNCS.DEM the demonstration file of
% the LaTeX macro package from Springer-Verlag
% for Lecture Notes in Computer Science,
% version 1.1 for LaTeX1e
%
\documentclass{llncs}
%
\usepackage{times}
%\usepackage{amsthm}
\usepackage{amsmath}
\usepackage{amssymb}
\usepackage[all]{xy}
\usepackage[english]{babel}
\usepackage{graphicx}
\usepackage{color}
\usepackage{subfigure}
\usepackage{here}

\usepackage{makeidx}  % allows for indexgeneration
%Claire
\newcommand{\ignore}[1]{}

\usepackage[applemac]{inputenc}


%------------------------------- New theorems

%------------------------------- macros, if any


\newcommand {\nat}{\mathbb{N}}
\newcommand {\scc}{\mathsf{scc}}
\newcommand {\ord}{\mathsf{ORD}}
\newcommand {\un}{\mathsf{un}}
\newcommand {\head}{\mathsf{head}}
\newcommand {\tail}{\mathsf{tail}}
\newcommand {\ad}{\mathsf{ad}}
\newcommand {\sub}{\mathsf{sub}}
\newcommand {\dom}{\mathsf{dom}}
\newcommand {\rank}{\mathsf{rank}}
\newcommand {\red}{\mathsf{red}}
\newcommand {\LGA}{\mathsf{LGA}}
\newcommand {\WAA}{\mathsf{WAA}}
\newcommand {\SAA}{\mathsf{SAA}}
\newcommand {\mybox}{\textrm{{\scriptsize $\,\Box\,$}}}

\newcommand{\dotcup}{\ensuremath{\mathaccent\cdot\cup}}




\begin{document}

%
\frontmatter          % for the preliminaries
%
\pagestyle{headings}  % switches on printing of running heads
\addtocmark{} % additional mark in the TOC
%
%
\mainmatter              % start of the contributions
%
\title{Deciding lower countable Wadge degrees of regular tree languages}
%
\titlerunning{Deciding lower countable Wadge degrees of regular tree languages}  % abbreviated title (for running head)
%                                     also used for the TOC unless
%                                     \toctitle is used
%

%
\author{
Alessandro Facchini\thanks{The author is supported by the \emph{Expressiveness of Modal Fixpoint Logics} project realized within the 5/2012 Homing Plus programme of the Foundation for Polish Science, co-financed by the European Union from the Regional Development Fund within the Operational Programme Innovative Economy (``Grants for Innovation'')} \and Henryk Michalewski\thanks{....}}
\institute{University of Warsaw \\ 
\email{\{A.Facchini,H.Michalewski\}@mimuw.edu.pl}}

\authorrunning{A. Facchini \and H. Mikalewski}   % abbreviated author list (for running head)

\maketitle              % typeset the title of the contribution


\begin{abstract}
\noindent blabla
\end{abstract}



%%%%%%%%%%%%%%%%%%%%%%%%
\section{Introduction}



%%%%%%%%%%%%%%%%%%%%%%%%%%
\section{Preliminaries}
\subsection{Trees and contexts}
Given a finite set $A$, a \emph{binary tree} over $A$ (from now on simply a tree) is a partial function $t: \{0,1\}^* \to A$ such that its domain $\dom(t)$ is prefix closed. A node of a tree $t$ is just an element $n \in \dom(t)$. A left child of a node $n$ of $t$ is the node $n.0$, while its right child is $n.1$. A leaf of a tree is a node without children. A tree $t$ over $A$ is infinite if it has no leaves (and thence it is a total function). We usually denote by $T_A$ the class of all  trees over $A$ and $T^\infty_A$ of all infinite trees over $A$. A set of trees over $A$ is called a tree language, or simply a language.

A \emph{multi-context} over $A$ is a tree $c$ over $A \cup \{\star\}$, where (i) $\star \notin A$, and (ii) $\star$ only labels some leaves of $c$ (at least one). A leaf of $c$ labelled by $\star$ is called a \emph{port}. Notice that a context can have infinitely many ports. For a multi-context $c$ and a function $\eta$ mapping each port of $c$ to a tree $t$ over $A$, by $c[\eta]$ we denote the tree given by substituting to the port $x$ in $c$ the tree $\eta(x)$. The class generated by $c$ and all possible mappings $\eta$ is denoted by $c[T_A]$, and by $c[T^\infty_A]$ we denote the class generated by $c$ and all possible mappings $\eta$ restricted to $T_A^\infty$. 

A finite multi-context is called a \emph{prefix}.
A multi-context with only one port is called a \emph{context}. %For an alphabet $A$, the class of all contexts over $A$ is denoted by $V_A$.
%% main text

\subsection{Topology}
For a finite alphabet $A$, we equip the class $T^\infty_A$ of all infinite trees over $A$ with the prefix topology. That is the basic open sets are sets of the form $p[T^\infty_A]$, for a prefix $p$ over $A$, and thus the open sets are of the form $\bigcup_{p \in P}p[T^\infty_A]$ for some set $P$ of prefixes. 

The class of Borel tree languages of $T^\infty_A$ is the
closure of the class of open sets of $T^\infty_A$ by countable unions and
complementation. Given $T^\infty_A$, the initial finite levels of the
Borel hierarchy are defined as follows with ${\bf\Sigma}^0_0(T^\infty_A)= \{\emptyset\}$ and
${\bf\Pi}^0_0(T^\infty_A)=\{T^\infty_A \}$.
\begin{itemize}
\item ${\bf\Sigma}^0_1(T^\infty_A)$ is the class of open subsets of $T^\infty_A$, 
\item ${\bf\Pi}^0_n(T^\infty_A)$ contains complements of sets from ${\bf\Sigma}^0_n(T^\infty_A)$, 
\item ${\bf\Sigma}^0_{n+1}(T^\infty_A)$ contains countable unions of sets from ${\bf\Pi}^0_n(T^\infty_A)$. 
\end{itemize}
%TODO remettre ? la fin si ca rentre
%By  convention $\Sigma^0_0(X)= \{\emptyset\}$ and $\Pi^0_0(X)=\{X \}$. 

%The classes defined above are closed under inverse images of continuous functions.  
%Given a class  $\mathcal{C}$, a set $U$ is called
%$\mathcal{C}$-hard if each set in $\mathcal{C}$ is an inverse image of $U$ under
%some continuous function. If additionally $U \in \mathcal{C}$, $U$ is said to be
%$\mathcal{C}$-complete.  

A much finer measure of the topological complexity of tree languages is the \emph{Wadge degree}.
If $L \subseteq T^\infty_A$ and $M\subseteq T^\infty_A$,   we say that $L$ is \emph{continuously (or Wadge)
reducible} to $M$, if there exists a continuous function $f$ such that $L=
f^{-1}(M)$. We write $L \leq_W M$ iff $L$ is continuously reducible to $M$.
Thus, given a certain Borel class $\mathcal{C}$, $M$ is $\mathcal{C}$-hard if $L
\leq_W M$ for every $L \in \mathcal{C}$.  This particular ordering is called the
{\em Wadge ordering}. If $L \leq_W M$ and $M \leq_W L$, then we write $L
\equiv_W M$. If $L \leq_W M$ but not $M \leq_W L$, then we write $L<_W
M$. The Wadge hierarchy is the partial order induced by $<_W$ on the
equivalence classes given by $\equiv_W$.

A language  $L$ is called {\em self dual} if it is equivalent to its complement, otherwise it is called {\em non self dual}.

\subsection{Algebra}

\section{Understanding the Complexity via Games}

\subsection{Topological Games}
Let $L$ and $M$ be two languages. The {\em Wadge game}
$\mathcal{W}(L, M)$ is an infinite two-player game (player I and player II) of perfect information defined as follows. During a play,
each player builds a tree, say $t_{I}$ and $t_{II}$. 
In each round both players add a finite number of children to the leaves
of their corresponding tree. Player I plays first and Player II is allowed to
skip his turn but not forever.  Player II wins the game iff $t_{I} \in L
\Leftrightarrow t_{II} \in M$.  Bill Wadge designed this game precisely in order
to obtain a characterisation of continuous reducibility.
\begin{lemma}[\cite{wadge}]
Let $L, M$ be two languages. Then  $L \leq_W M$ iff Player II has a winning strategy in the game $\mathcal{W}(L, M)$.
\end{lemma}

An ordinal number is the order type of a well-ordered set. 
%The finite ordinals are the natural numbers ($0, 1, 2, \dots$). 
The least infinite ordinal is denoted by $\omega$ and corresponds to the order-type of the set of all natural numbers. We say that an ordinal $\alpha$ is countable if there is a bijection between $\alpha$ and $\omega$. The first uncountable ordinal is denoted by $\omega_1$.
A subset $B$ of an ordinal $\alpha$ is said to be \emph{cofinal} if 
for every $a \in \alpha$ there exists some $b \in B$ such that $a \in b$.
The \emph{cofinality} of an ordinal $\alpha$ is thence the smallest ordinal $\beta$ that is the order type of a cofinal subset of $\alpha$.

  From Borel determinacy,
if both $L$ and $M$ are Borel, then $\mathcal{W}(L, M)$ is determined.
As a consequence, a variant of Martin-Monk's result shows that $<_W$ is
well-founded. The \emph{Wadge degree} for sets of finite Borel rank is
inductively defined by:
\begin{itemize}
\item $d_W(\emptyset)=d_W(\emptyset^\complement)=1$,
\item $d_W(L)=\sup\{d_W(K)+1\colon K \text{ non self dual}, K <_WL\}$ for $L>_W\emptyset$.
\end{itemize}

For instance, open, non-closed sets have degree 2, just like closed,
non-open sets. All clopens have degree 1. 
Let $\exp(\alpha) = \omega_1^\alpha$, and let 
$\,{}^{\omega_1} \epsilon_0 = \sup_{n \in\omega} \exp^n(\omega_1)$, the least fixpoint of the 
ordinal exponentiation of base $\omega_1$. This is known to be the
height of the Wadge hierarchy of all tree languages (recognizable or
not) of finite Borel rank. More precisely, if $L$ is ${\bf
  \Sigma}^0_n$-complete for $n>1$, then $d_W(L) = \exp^{n-1}(1)$ for $n>1$  (cf. \cite{dup1}).

For each degree there are exactly three equivalence
classes with the same degree, represented by $U$, $U^\complement$ and
$U^\pm = \{t \bigm | t(\epsilon) = a, t.0\in U \} \cup  \{t \bigm |
t(\epsilon) \neq a, t.0\notin U \}$ for some non self-dual set $U$ and
$a\in\Sigma$. It easy to check that $U, U^\complement <_W U^\pm$ and
$U^\pm$ is self-dual.

For each non self-dual set one can determine its sign, $+$
or $-$, which specifies precisely the $\equiv_W$-class
(cf. \cite{dup1}). For sets $U\subseteq T_\Sigma$ with $d_W(U)$ of countable cofinality, the sign
is $+$ if $U$ is Wadge equivalent to the set of trees over $\Sigma
\cup \{c\}$, $c\notin \Sigma$, which have no $c$ on the leftmost
branch, or the first $c$ is in the node $0^i$ and $t.0^i \in U$. The
sign is $-$ if $U$ is equivalent to the complement of this set. 
For instance, $\emptyset$ and open, non-closed sets have sign -,
while the whole space and closed, non-open sets have sign $+$.
For sets of cofinality $\omega_1$, the definition is more complicated, but
${\bf \Sigma}^0_n$-complete sets have sign $-$, and ${\bf
  \Pi}^0_n$-complete sets have sign $+$. All self-dual sets by
definition have sign $\pm$. Thus an ordinal $\alpha <\,
{}^{\omega_1}\epsilon_0$ and a sign $\epsilon \in \{+,-,\pm\}$,
determine a $\equiv_W$-class, denoted  $[\alpha]^\epsilon$. 

\subsection{Set-Theoretical Operations}

We start defining four basic operations on conciliatory sets of trees. Let $L, M \subseteq T^{\sf c}_\Sigma$, and assume that $\Sigma$ contains at least two letters, $a$ and $b$. Define
{\em alternative} ($\lor$), {\em parallel composition}  ($\land$), {\em disjunctive product} ($\diamond$), and {\em
conjunctive product} ($\mybox$) as
\begin{align*}
L \lor M = \,&\{t\colon t(\varepsilon) = a\,, t.0 \in L \text{ and } t.1 \in M \textrm{, or } t(\varepsilon) \neq a\}\,,\\
L \land M = \,&\{t\colon t(\varepsilon) = a\,, t.0 \in L \textrm{ or } t(\varepsilon) \neq a\,, t.0 \in M\}\,,\\
L \diamond M =\, &\{t\colon  t.0 \in L \textrm{ or } t.1 \in M\}\,,\\
L \mybox M =\, &\{t\colon t.0 \in L \textrm{ and } t.1 \in M\}\,.
\end{align*}
Multifold alternatives and parallel compositions are performed from left to right, e.g.,  $L_1 \lor L_2 \lor L_3 \lor L_4 = (((L_1 \lor L_2) \lor L_3) \lor A_4 )$. It is easy to see that these four operations define associative and commutative operations on Wadge equivalence classes. The symbol $(L)^n$ denotes $\underbrace{L \land \ldots \land L}_{n}$.\label{automatapower}



Another useful operation on sets is the following. Let $L, M \subseteq T^{\sf c}_\Sigma$. We define
 the set $L \to M$ as the set of trees $t \in T^{\sf c}_{\Sigma\cup \{a\}}$, with $a \notin \Sigma$, satisfying any of the following conditions:

\begin{itemize}
\item $t.0 \in L$ and $a = t(10^n)$ for all $n$,
\item $10^{n}$ is the first node on the path $10^*$ such that $a \neq t(10^{n})$ and $t.10^{n}0 \in M$.
\end{itemize}
A player in charge of $L \to M$ is like a player in charge of $L$ endowed with an extra move, which can be used only once, that erases everything played before. Then she can restart the play being in charge of $M$.  We say that  a non-self dual set $L\subseteq T_\Sigma$ is \emph{initializable} when $L \geq_W L \to L$.


\begin{Proposition}[\cite{dup3,dup95}]\label{p_2a}
{\bf (PD)}
Let $L, M\subseteq T^{\sf c}_\Sigma$ be two non self dual sets, 
\begin{enumerate}
\item Assume that $M$ is initializable, that $L <_W M $ and that for every initializable set of trees $C$, if $L<_WC$, then $M \equiv_W C$ or $M^\complement \equiv_W C$. Then it holds that  $d_W(M)=d_W(L)\centerdot \omega_1$.
\item Any set of Wadge degree $d_W(L)\centerdot \omega_1$ is initializable.
\end{enumerate}
\end{Proposition}
These properties of initializable sets will be very useful.

\vspace{0.2cm}
\noindent {\bf Sum and supremum :} \hspace{0.1cm}
%%%addition
Suppose that $L, M \subseteq T^{\sf c}_{\Sigma} $. We define the set $M+ L$ as $L \to M \lor M^\complement$. 
%This set is weakly recognizable (\cite{dm7}). The weak alternating automaton recognizing it is denoted by $B + A$.
From the point of view of the player in charge of the set $M+ L$  in a Wadge Game, everything goes as if she was starting the game being in charge of $L$. So, provided she plays in such a way that $a$ always holds in the rightmoust branch of the tree, the question whether the resulting infinite tree she will have produced at the end of the run belongs to $M+ L$  or not reduces to the question whether the tree starting from  the left son of the root belongs to $L$ or not. But at any moment of the run she can play a node $11^n$ not labelled with $a$. Then, everything looks like the whole (finite) tree played since the beginning of the game is erased and he is now in charge of: $M$ if $a$ is the label of the node $(11^n1)$, $M^\complement$ else. 

The following remark ensures that the set-theoretical operation $+$ is well-behaved and in particular is the counterpart of the ordinal sum on Wadge degrees.
\begin{remark}[\cite{dup3,dup95}]{\bf (PD)}
\label{r_sum}
Let $L, M, L', M'$ be four non self-dual sets of conciliatory binary trees. Then

\begin{itemize}
\item  $ (L + M)^\complement \equiv_W  L+ M^\complement $,
\item The operation $+$ preserves the Wadge ordering: \[ \text{if } L' \leq_W L \text{ and } M' \leq_W M \text{ then } L'+ M' \leq_W L+ M
\]
\item $d_W(L+ M)= d_W(L) + d_W(M)$.
\end{itemize}
\end{remark}
\noindent As for alternative, it is easy to see that sum defines associative and commutative operations on Wadge equivalence classes. 

The next operation is a generalization of $\lor$ and $+$. Let $\lambda < \omega_1$, and $L_\kappa \subseteq T^{\sf c}_{\Sigma \cup \{b\}}$
for any $\kappa<\lambda$. Fix any $1-1$ map $f: \omega \to \lambda$. Thus, define $\mathrm{sup}^-_{\kappa<\lambda} L_\kappa$ as the set of trees $t
\in  T^{\sf c}_{\Sigma \cup \{b\}}$ satisfying the following conditions for some $k$:
\begin{itemize}
\item $0^k$ is the first node on $1^*$ labeled with $b$,
\item $t.0^k1 \in L_{f(k)}$.
\end{itemize}
Intuitively, a player in charge of $\sup^-_{\kappa<\lambda} L_\kappa$ is given the choice
between the $L_\kappa$'s. The decision is determined by the number of labels different from $b$ played on
the leftmost branch of the tree before the first $b$. If the player keeps not
playing $b$  forever on the leftmost branch, the tree will be rejected.  

Define also
$\sup^+ _{\kappa<\lambda} L_\kappa$ as $\sup^-_{\kappa<\lambda} L_\kappa\cup \{t :\, \forall_n\; t(1^n)\neq b\}$.
The difference from the previous operation is that now, when the
player does not plays $b$ on the leftmost branch, the obtained tree is
accepted. Note that the operations are dual: 
\[\left ( \sup^+ _{\kappa<\lambda} L_\kappa \right )^\complement = \sup^-_{\kappa<\lambda} \left ( L_\kappa^\complement
\right )\]
The following remark ensures that the set-theoretical $\sup$ preserves the Wadge order and is the counterpart of the ordinal supremum on Wadge degrees.
\begin{remark}[\cite{dup3,dup95}]{\bf (PD)}
\label{r_sup}
Let $(L_\kappa)_{\kappa<\lambda}$ and $(M_\kappa)_{\kappa<\lambda}$ be two countable familiies of non self-dual sets of conciliatory binary trees. Then
\begin{itemize}
\item if for all $\kappa \in \lambda$, $L_\kappa \leq_W M_\kappa$ holds, 
then $\sup^+_{\kappa<\lambda} L_\kappa \leq_W \sup^+_{\kappa<\lambda} M_\kappa$ and $\sup^-_{\kappa<\lambda} L_\kappa \leq_W \sup^-_{\kappa<\lambda} M_\kappa$
 hold too,
\item$d_W(\sup^+ _{\kappa<\lambda} L_\kappa)= d_W(\sup^- _{\kappa<\lambda} L_\kappa)=\sup_{\kappa<\lambda}d_W( L_\kappa)$.
\end{itemize}
\end{remark}




%%%%%%MULTIPLIC
\vspace{0.2cm}
\noindent {\bf Countable multiplication :} \hspace{0.1cm}
With the help of ordinal sum and countable supremum we easily define the set-theoretic counterpart of the countable multiplication as an iterated sum. 
Let $L \subseteq  T^{\sf c}_{\Sigma} $.  Inductively we define: 
\begin{itemize}
\item $L \bullet 1 = L$, 
\item $L \bullet (\alpha + 1) = (L \bullet \alpha)+L$, 
\item $L \bullet \lambda = \sup^+_{\kappa \in \lambda} L \bullet \kappa$ when $\lambda$  is some limit countable ordinal.
\end{itemize}

As for the previous operations, we remark that   the set-theoretical coutable multiplication preserves is the counterpart of the ordinal operation of coutable product  on Wadge degrees.

\begin{remark}[\cite{dup3,dup95}]{\bf (PD)}
\label{r_mult}
Let $L, M\subseteq T^{\sf c}_\Sigma$ be two non self dual sets. Then for every countable ordinal $\lambda$
\begin{itemize}
\item $(L\bullet \lambda)^\complement \equiv_W L^\complement \bullet \lambda$,
\item if $L \leq_W M\text{ then } L\bullet \lambda \leq_W M\bullet \lambda$
\item $d_W(L \bullet \lambda)= d_W(L) \centerdot \lambda$
\end{itemize}
\end{remark}


From the player's point of view when involved in Wadge Games, being in charge of a set of the form $L\bullet \lambda$ is like a player being in charge of $L$ with the additional option to restart the run at any moment being 
in charge of its complement $L^\complement$ instead of $L$ and start again and again replacing alternatively $L^\complement$ by $L$ and  $L$ by 
$L^\complement$,  provided that at every such changing the player decreases the ordinal $\lambda$.  Therefore, during the run, this procedure will produce a decreasing finite sequence of ordinals, preventing her from initializing the game indefinitely\footnote{ 
Note that the multiplication by $\omega_1$ is reached  by allowing such a process to be infinite. However we will not discuss this operation in this paper.}.



%%%%%%%%%
%CUTTING
%%%%%%%%%

\subsection{Cutting games}

%\subsection{Delayed cutting games}
We are going now to introduce another type of two-player game of perfect information. 

Let $( L_i : i \in \omega )$ be a sequence of tree languages over $\Sigma$, and $p$ a prefix over $\Sigma$. 

The \emph{simple} cutting game over $( L_i : i \in \omega )$ and a prefix $p$ of length $k$, denoted by $\mathcal{X}^p_k(L_i : i \in \omega )$ is played by two players, Constrainer and Alternator.

At the beginning of game, Constrainer chooses $p$.
For each $i \in \{1, \dots, k \}$, round $i$ of the game is played as follows.
\begin{itemize}
\item Alternator chooses a tree $t_i \in L_i$ extending the choice of Constrainer, 
\item Constrainer chooses a prefix of the choice of Alternator.
\end{itemize}
If Alternator cannot move, she loses, if she manages to survive $k$ rounds, then she wins.

The infinite cutting game over $( L_i : i \in \omega )$ and a prefix $p$ of length $k$, denoted by $\mathcal{X}^p_\infty(L_i : i \in \omega )$, is played just like a simple game but without the restriction to a fixed given number of round. Alternator wins iff she can always move.

The $M^+$-delayed cutting game over $( L_i : i \in \omega )$, denoted by $\mathcal{X}^{M^+}_\omega(L_i : i \in \omega )$, is like a simple cutting game, except that a mini game is played to determine the prefix $p$ and the length $k$ of the match. 
The mini game goes as follows. Firstly, Alternator chooses a tree $t \in M$. Then Constrainer chooses a prefix $p$ of $t$ and a finite ordinal $k$. Finally the two players starts to play the simple cutting game $\mathcal{X}^p_k(L_i : i \in \omega )$. 
The $M^-$-delayed cutting game over $( L_i : i \in \omega )$ is played like a $M^+$-delayed cutting game over $( L_i : i \in \omega )$ except that at first Alternator chooses a tree $t \notin M$.

When $L_{2i}=L$ and $L_{2i +1}=L^\complement$, then we simply write $\mathcal{X}^p_k(L, L^\complement), \mathcal{X}^p_\infty(L, L^\complement)$ and  $\mathcal{X}^{M^+}_\omega(L, L^\complement)$.

\begin{proposition}\label{prop:omega} Let $L$ be a tree language.

$[\omega]^+ \leq_W L$ iff Alternator has a winning strategy in $\mathcal{X}^{L^+}_\omega(L^\complement, L)$. 

\end{proposition}
\begin{proof}
For the direction from left to right, we reason as follows. Let $f$ be the winning strategy for Player two in $\mathcal{G}([\omega]^+, L)$. As a first move, Alt plays the tree $t$ given by applying the strategy $f$ against Pl I playing always accepting. Now, suppose that Const plays a prefix $p$ of depth $\ell$ and $k$. Thus Alternator looks at the shadow game used to determine $t$, but at $k+1$ round, she makes Player I erasing his game and decide to play into $[k']^-$, for a $k' \geq k + \ell +1$, and appply the winning strategy $f$ in the game that continues with Player 1 playing rejecting. Assume the obtained tree is $t'$. We have that:
\begin{itemize}
\item $t' \notin L$ and $t'$ extends $p$,
\end{itemize}
thus it is an admissible move for Alt.
Now, assume at next round Constrainer chooses a prefix $p_1$ (extending $p$, wlog) of $t'$, whose depth is $\ell_1$. Then in the shadow match, Alt change Play 1 moves in the sense that at turn $k_1 \geq k' + \ell_1$ he decrease the ordinal of one and starts to play accepting. The tree so obtained by $f$ when Pl 1 keeps playing accepting is next Alternator moves. For the same reasons as before, such a move is admissible. By continuing such a strategy, it is clear that Alternator wins.

For the direction from right to left, we describe a winning strategy for Player 2 in $\mathcal{G}([\omega]^+, L)$ as follows. In the back she keep tracks of a shadow match in $\mathcal{X}^{L^+}_\omega(L, L^\complement)$ where she applies the  winning strategy for Alternator. 
As long as Pl 1 plays accepting, Pl 2 just plays the initial choice of Alternator. Now, assume that at round $n$, Pl 2 decides to decrease his ordinal to $k$ and to play rejecting, then in the shadow match, Pl 2 makes Constrainer play a prefix of depth $n-1$ and the ordinal $k+1$, then Alt plays the tree $t_1$ obtained by applying her winning strategy. So Pl starts to play into $t_1$. By construction $t_1$ is not in $L$, thus if Pl1 continues to play rejecting, she wins. If at round $n_1> n$ Pl 2 decrease his ordinal of one and decides to start to play accepting, then in the second round of the shadow match, Pl 2 makes Constrainer choosing a prefix of depth $n_1$, and Alternator apply her winning strategy to obtain a tree $t_2$ extending the choosen prefix of $t_1$. Thus Pl 2 just start to play into $t_2$. Since this tree is in $L$, if Pl 1 keeps playing accepting, he looses. By continuing of playing with such a strategy, Pl 2 is ensured to win. 
\end{proof}

\begin{proposition}\label{prop:infinity} Let $L$ be a tree language. For every prefix $p$,

$d_W([p]^{-1}L) \geq \omega_1$ iff Alternator has a winning strategy in $\mathcal{X}^p_\infty(L, L^\complement)$. 

\end{proposition}
\begin{proof}
For the direction from left to right, it is enough to apply the fact that $d_W([p]^{-1}L) \geq \omega_1$ then Player II has a winning strategy in $\mathcal{G}(M, [p]^{-1}L)$, where $M$ is either $[\omega_1]^+$ or $[\omega_1]^-$, 
where, from the point of view of Wadge games, a player in charge of $[\omega_1]^+$ 
is like a player in charge of the $\emptyset$, with at every point the possibility of restarting anew and being in charge its complement, with the condition that if she restart the game infinitely often, she is rejecting.
%
% that Alternator has a winning strategy in $\mathcal{X}^\varepsilon_\infty(L, L^\complement)$, where $L$ is the canonical set of Wadge degree $\omega_1$ over $\{a,b\}$ defined by the property ``all nodes on $1^\omega$ are labelled $a$, and there is a $k$ such that the word induced by $1^k0^\omega$ is in $a^k b^* a^\omega$. But 


The other direction is proved by showing that for every countable ordinal $\kappa$, Player II has a winning strategy in $\mathcal{G}([\kappa]^+, [p]^{-1}L)$ and $\mathcal{G}([\kappa]^-, [p]^{-1}L)$.
% induced by Alternator's winning strategy in $\mathcal{X}^p_\infty(L, L^\complement)$. 
%{\tt TODO: say explicitly what ``induced'' here mean.}

This is done by induction on countable Wadge degrees.
Let $\kappa=1$. Then the claim is trivially proved.
Assume now that $\kappa= \beta + 1$. We first consider the case for $+$. At first Player II applies the winning strategy in $\mathcal{G}([1]^+, [p]^{-1}L)$. Now assume that at round $n$ Player 1 decides the erase everything and being in charge of $[\beta]^-$ (the case for $[\beta]^+$ is exactly the same). Assume that before her turn at round $n$, Player 2 has played $p'$. From round $n+1$ she just apply the winning strategy given by the induction hypothesis in $\mathcal{G}([\beta]^-, [p']^{-1}L)$. Now, let's consider the case for $-$. At first Player II applies the winning strategy in $\mathcal{G}([1]^+, [p]^{-1}L)$ by following the strategy induced by the following shadow match in $\mathcal{X}^p_\infty(L, L^\complement)$: at the first round alternator play any tree of $L$ extending $p$, then Constrainer plays $p$, and Alternator plays a tree $t$ with prefix $p$ outisde $L$. The induced strategy in $\mathcal{G}([1]^+, [p]^{-1}L)$ for Pl II is thence to play $t$. 
Assume that at round $n$ Player I decides the erase everything and being in charge of $[\beta]^-$ (the case for $[\beta]^+$ is exactly the same). Assume that before her turn at round $n$, Player II has played $p'$. From round $n+1$ she just apply the winning strategy given by the induction hypothesis in $\mathcal{G}([\beta]^-, [p']^{-1}L)$.
%{\tt TODO: finish here, clear idea by using the fact of keeping track of the strategy.}
\end{proof}

%%%%%%%%%%%%%%%%%%%%

\section{A characterization of regular languages  below $\omega_1$}

We recall that, given a language $L$, the set $H_L$ is the set of tree types, and the set $V_L$ of context types of $L$. Both are finite when $L$ is regular. Following \cite{bp}, we use  games on types, either finite $\mathcal{X}(h_1, \dots, h_n)$ or infinite $\mathcal{X}(h_1, \dots, h_n, \dots)$ - same for context types-, for a given a language $L$.



Thus, 
$\mathcal{H}_L:= \{(h_1, \dots, h_n) 
\mid 
n \in \omega 
$
and alternator wins $\mathcal{X}(h_1, \dots, h_n)\}$
and 
$\mathcal{V}_L:= \{(v_1, \dots, v_n) \mid n \in \omega 
$
and alternator wins $\mathcal{X}(v_1, \dots, v_n)\}$.

Sets 
$\mathcal{H}^\infty_L$ and $\mathcal{V}^\infty_L$ are defined analogously.


%For a multicontext $c$, we can also define the game $\mathcal{X}^c(h_1, \dots, h_n, \dots)$
%the same way as $\mathcal{X}(h_1, \dots, h_n, \dots)$ with the added restriction that the first tree played by alternator must extend the multicontext $c$. 


A type tree for $L$ is a tree over $\mathcal{H}_L$. For a given tree $t$, there is a type tree $\sigma_t$ induced by $t$ such that for every node $n \in \dom(\sigma_t)$, $\sigma_t(n)$ is the type of the tree $t.n$.

Let $\sigma$ be a type tree, and $t$ a tree. $\sigma$ is locally consistent with $t$ if $\dom(\sigma)=\dom(t)$ and for every node $n \in \dom(t)$ such that $t(n)=a$, 
\begin{itemize}
\item if $n$ is a leaf, then $\sigma(n)$ is the type of $a$,
\item if $n$ has two children $m_\ell$ and $m_r$, then $\sigma(n)$ is the the obtained by applying $a$ to the pair $(\sigma(m_\ell), \sigma(m_r))$.
\end{itemize}

A finite strategy tree is a tuple

\[ \mathfrak{s}=(t, \sigma_1, \dots, \sigma_k) \] where
\begin{itemize}
\item $t$ is a tree, the support of the strategy
\item $\sigma_t=\sigma_1$, and
\item $\sigma_\ell$ is locally consistent with $t$, $\ell \leq k$,
\item for each $n \in \dom(t)$, $(\sigma_1(n), \dots, \sigma_k(n)) \in \mathcal{H}_L$.
\end{itemize}

A  infinite strategy tree is defined analogously.

Given a prefix $p$, we can associate a multi context $c_p$ given by adding two holes to every leaf of $p$. For a context $c$, the set $Holes(c)$ is the set of all holes of $c$. The following Lemma follows immediately by definition of a strategy tree.

\begin{lemma}\label{lemma:short_strategy}
Let $\mathfrak{s}=(t, \sigma_1, \dots, \sigma_k)$ be a any strategy tree. Given the  game $\mathcal{X}(\sigma_1(\varepsilon), \dots, \sigma_k(\varepsilon))$ and a strategy for Constrainer given by always cutting at level $i$, Alternator wins  by playing as follows:
\begin{itemize}
\item at first, Alternator plays $t$, then
\item for each hole $n$ at level $i$ of the multi context given by Constrainer's move, Alternator plugs in the tree given by her winnings strategy $\mathcal{X}(\sigma_1(n), \dots, \sigma_k(n))$.
\end{itemize}
In particular, if from a certain $j<k$ on $\sigma_\ell(n)=\sigma_{\ell+1}(n)$, $j\leq \ell < k$, then for each round $j< \ell < k$ she always plugs in the same tree of type $\sigma_j(n)$ chosen at round $j$.

\end{lemma}

Let $\mathfrak{s}$ be a finite strategy tree. The root alternation of $\mathfrak{s}$ is the alternation of the root sequence, while the limit of alternation of $\mathfrak{s}$ is the maximal number $k$ such that infinitely many subtrees of $\mathfrak{s}$ have root alternation $k$.

We say that $L$ has bounded root alternation if the set $\mathfrak{S}_L$ of finite strategy trees has bounded root alternation. Analogously for limit alternation.

The following proposition is a straightforward generalization of an analogous proposition (5.2) in \cite{bp}

\begin{proposition}\label{prop:tree_to_types}
 For a regular language $L$ of infinite trees and any prefix $p$, the following conditions are equivalent.
 \begin{itemize}
\item Alternator wins the game $\mathcal{X}^p_\infty(L, L^\complement)$ 
 \item  There are tree types $h, g$, such that $h\neq g$ and Alternator wins $\mathcal{X}^p_\infty(h, g)$.
 \end{itemize}
\end{proposition}

The following Lemma will turn out to be useful:

\begin{lemma}[Appendix of \cite{bp}]\label{lemma:limit}
Let $(t_n: n <\omega)$ be a sequence of trees that converges to $t^*$ and let $(\sigma_n: n \in \omega)$ be a sequence of type trees that converges to $\sigma^*$. If $\sigma_n$ is locally consistent with $t_n$ for every n, then $\sigma^*$ is locally consistent with $t^*$.
\end{lemma}

Everything now is ready to prove the main result of this note.

\begin{theorem}
Let $L$ be a regular tree language. The following conditions are equivalent:
\begin{enumerate}



%\item The following identity is not satisfied: $(u_2w_2^\omega v)^\omega u_1w_1^\infty = (u_2w_2^\omega v)^\infty$ if $(u_1, u_2) \in V_L$ and $(w_1, w_2) \in V_L$
\item $L$ has unbounded limit alternation
\item The strategy graph $G_L$ is recursive
\item $d_W(L) \geq \omega_1$



\end{enumerate}
\end{theorem}

\begin{proof}
The implication $(1) \Rightarrow (2)$ is proved in \cite{bp}.  
We show that $(2) \Rightarrow (3)$. Assume the strategy graph is recursive. This means that there exists a strongly connected component that contains two nodes $(v, h)$ and $(v', h')$ with $h \neq h'$. 
Now, if there exists a path between $(v, h)$ and $(v', h')$, there is also an edge between $(v, h)$ and $(v', h')$. 
Moreover, for every sequence $((v_i,h_i) : i \in \alpha)$, for $\alpha \leq \omega$, if there is an edge from $(v_i, h_i)$ to $(v_{i+1}, h_{i+1})$, this means that Alternator has a winning strategy in $\mathcal{X}(h_i: i \in \alpha)$. So, take $(v_i, h_i)= (v,h)$ for $i$ even, and $(v_i, h_i)= (v',h')$ for $i$ odd. We have that Alternator has a winning strategy in $\mathcal{X}^\varepsilon_\infty(h, h')$. By Proposition \ref{prop:tree_to_types}
Alternator has a winning strategy in  $\mathcal{X}^\varepsilon_\infty(L, L^\complement)$.


Now we turn to the implication  $(3) \Rightarrow (1)$. 
By proposition \ref{prop:infinity} it is enough to verify that 
 if Alternator has a winning strategy in $\mathcal{X}^\varepsilon_\infty(L, L^\complement)$ then $L$ has unbounded limit alternation.
 
Assume Alternator has a winning strategy $f$ in  $\mathcal{X}^\varepsilon_\infty(L, L^\complement)$. We are going to verify that there is a infinite strategy tree $\mathfrak{s}^\infty$ with infinite root alternation. The strategy tree is constructed as follows. First of all, $f$ can be represented as a tree $t_f$: 
\begin{itemize}
\item the root is labelled by $\varepsilon$, and its unique child is labelled by Alternator's move given by $f(\varepsilon)$,
\item if a node $n$ at odd depth (and thus representing Alternator's move) is labelled by a tree $t_n$, then for every prefix $p$ of $t_n$, there is a unique child $n$ labelled by $p$,
\item if a node $n$ at even depth (and thus representing Spoiler's move) is labelled by a prefix $p$ of the tree $t_f(m)$, which is the label of the unique ancestor $m$ of $n$, then there is a unique child $n'$ of $n$ labelled by $f(t_f(m_1)\dots t_f(m_k))$, where $(m_1\dots m_k)$ is the path from the root of the tree to $m$.
\end{itemize}
Thus from now on we identify $f$ and $t_f$.

Given a node $n$ of $f$ labelled by a prefix $p$, and $\alpha < \omega$,  by $(f.n)|_k$, we denote the induced  winning strategy  for Alternator in  $\mathcal{X}^p_\alpha(L, L^\complement)$, if $n$ is at depth $2i$ with $i$ even, and in  $\mathcal{X}^p_\alpha(L^\complement, L)$, if $n$ is at depth $2i$ with $i$ odd.

Now, if we prove the following claim we are done:
\begin{claim}\label{claim:strategy}
For every node $n$ of $f$ labelled by a prefix $p$, 
there is a sequence of strategy trees $\mathfrak{S}_n=(\mathfrak{s}^n_i: i < \omega)$ such that for each $n$
\begin{enumerate}
\item $\mathfrak{s}^n_\ell=(t, \sigma_1, \dots, \sigma_\ell)$, with $\sigma_{2k+1}$ the type of $L$ and $\sigma_{2k}$ the type of  $L^\complement$ if $n$ is at depth $2i$ with $i$ even, else dually,
\item $\mathfrak{s}_{n+1}$ extends $\mathfrak{s}_n$,
\end{enumerate}
\end{claim}
Indeed, given the claim above, from point 1 we have that
for each node $n$ labelled by a prefix $p$, and each $k < \omega$, $\mathfrak{s}^n_k$ has root alternation $k$ and defines a winning strategy for Alternator in $\mathcal{X}^p_k(L, L^\complement)$ if $n$ is at depth $2i$ with $i$ even, in $\mathcal{X}^p_k(L^\complement, L)$ else.
From point 2, it is thus enough to take $\mathfrak{s}^\infty$ has the limit of the sequence $(\mathfrak{s}^\varepsilon_k: k < \omega)$, and $\mathfrak{S}=\bigcup_{k \in \omega} \mathfrak{s}^\varepsilon_k$. 
Now, assume limit alternation of $\mathfrak{S}$ is bounded.
This means that there is a strategy tree $\mathfrak{s} \in \mathfrak{S}$, and there are $k, k', i, \ell, $ with $k < k'$ such that
 there is a set $\{n_1, \dots, n_\ell\} \subset \dom(t)|_i$, where $\dom(t)|_i$ denotes the set of nodes of $t$ of depth at most $i$, and
\begin{itemize}
%\item there is a node $n^* \in \{n_1, \dots, n_\ell\}$ with two children $m_1, m_2 \in \dom(t)|_i \setminus \{n_1, \dots, n_\ell\}$
\item $\sigma_k(n_i)\neq \sigma_{k'}(n_i)$ for $i=1, \dots, \ell$,
\item $\sigma_k(n)= \sigma_{k'}(n)$, for every $n \in \dom(t)|_i \setminus \{n_1, \dots, n_\ell\}$.
\end{itemize}
Assume $\mathfrak{s}=(t, \sigma_1, \dots, \sigma_j)$, and let us consider the game $\mathcal{X}( \sigma_1(\varepsilon), \dots, \sigma_j(\varepsilon))$ where at first Alternator plays $t$ and then Constrainer uses the strategy given by cutting always at level $i+1$. By Lemma \ref{lemma:short_strategy}, the trees played at round $k$ and $k'$ using her winning strategy are the same, say $t'$. But by locally consistency there is a node $n$ such that the subtree $t'.n$ of $t'$ with root in $n$ has both type $\sigma_k(n)$ and $\sigma_{k'}(n)$, with $\sigma_k(n) \neq \sigma_{k'}(n)$. Contradiction.



We prove the claim \ref{claim:strategy} by verifying by induction that the sequence $(\mathfrak{s}^n_i: i < k)$ can be extended to a sequence $(\mathfrak{s}^n_i: i < k+1)$ satisfying the properties 1 and 2 of the claim. 



In order to do so, we reason as follows. First of all, given a sequence of types $(\sigma_i : i \in \omega)$, by compactness there is a converging subsequent whose limit is  $\sigma^*$. We assume such a subsequence given: every time we are going to choose the limit of a converging  subsequence of the sequence $(\sigma_i : i \in \omega)$, we are going to take $\sigma^*$.

Now, for the initial step, it is enough to take for each node $n$ and prefix label $p$, $\mathfrak{s}^n_1=(t, \sigma_1)$, where $t$ is given by applying $f$ to the considered position, and $\sigma_1$ is the unique tree type induced by $t$. 

We now describe the general construction for $k>1$. Fix any node $n$ with label a prefix $p(n)$. Assume the answer given by $f$ is $t$. For every prefix $p$ of $t$, we have a strategy tree $\mathfrak{s}^m_{k-1}=(t_p, \sigma_{p2}, \dots, \sigma_{pk})$, where $m$ is the node corresponding to $p$ in $f$. 
By compactness, there is an infinite subsequence of prefixes $(p_i: i < \omega)$ whose limit is $t$ and such that all of the sequences
$(t_{p_i}: i <\omega ), (\sigma_{p_i2}: i <\omega ), \dots,  (\sigma_{p_ik}: i <\omega )$ are convergent. Let the limits of these sequences be $t, \sigma^*_2, \dots, \sigma^*_k$.
For each $p$, the type trees $(\sigma_{p2}, \dots, \sigma_{pk})$
are locally consistent with $t_p$. Therefore, by Lemma \ref{lemma:limit} it follows that the limits $\sigma^*_2, \dots, \sigma^*_k$ are locally consistent with $t$. Finally, define $\sigma^*_1$ to be the unique type tree induced by  $t$ (globally consistent with it). We have just proved that
$(t, \sigma^*_1, \dots, \sigma^*_k)$
is a strategy tree. Because root values are preserved under limits, the root value
of this strategy tree is of the desired kind. By induction hypothesis, such a construction has the property that $\mathfrak{s}^m_{k+1}$ extends $\mathfrak{s}^m_{k}$. This concludes the proof of the claim.
  \end{proof}






\section{Conclusion}


%\subsubsection*{Acknowledgment.}  

%%%%%%%%%%%%%%%%


\begin{thebibliography}{00}

%\begin{thebibliography}{dupfacmur}

\bibitem{bp}
\textsc{Bojanczyk} M., \textsc{Place} T. :  Regular Languages of Infinite Trees That Are Boolean Combinations of Open Sets. ICALP (2) 2012: 104-115


%\bibitem{dup95}
%\textsc{Duparc} J. : The Normal Form of Borel Sets of Finite Rank. PhD Thesis (in french),  University of Paris 7 (1995)

\bibitem{dup1}
\textsc{Duparc} J. : Wadge Hierarchy and Veblen Hierarchy Part 1: Borel Sets of Finite Rank. \textit{Journal of Symbolic Logic} 66(1): 56-86 (2001)

%\bibitem{dupinf}
%\textsc{Duparc} J. : Wadge Hierarchy and Veblen Hierarchy Part 2: Borel Sets of Infinite Rank. \textit{Submitted}

\bibitem{dup3} 
\textsc{Duparc}, J.: 
A Hierarchy of Deterministic Context-Free $\omega$-languages.
\textit{theoretical Computer Science} 290: 1253--1300 (2003)


%\bibitem{dm7}
%\textsc{Duparc} J., \textsc{Murlak} F. : On the Topological Complexity of Weakly Recognizable Tree Languages. \textit{FCT 2007}, LNCS 4639: 261-273 (2007)



\bibitem{kechris}
\textsc{Kechris} A.: \emph{Classical Descriptive Set Theory}. Springer (1995)


\bibitem{wadge}
{\sc Wadge} W.W.: {\em Reducibility and Determinateness on the Baire Space.}  Ph.D. Thesis, Berkeley (1984).



\end{thebibliography}
\end{document}
